%!TEX TS-program = Arara
% arara: lualatex
% arara: biber
% arara: lualatex
% arara: lualatex

\documentclass[12pt,ngerman]{scrbook}

\usepackage[utf8]{inputenc}
\usepackage[T1]{fontenc}
\usepackage{booktabs}
\usepackage{babel}
\usepackage{graphicx}
\usepackage{paralist}
\usepackage{xcolor}

\usepackage{listings}
\usepackage{libertine}
%\usepackage{palatino}
%\usepackage[scale=0.85]{sourcecodepro}
\usepackage{blindtext}

\setlength{\parindent}{0pt}
\setlength{\parskip}{1em}

\usepackage{xcolor}
\usepackage{mdframed}
\usepackage{tikz}

\usepackage[headsepline=0.5pt,footsepline=0.0pt]{scrlayer-scrpage}
%\usepackage[left=2cm,right=4cm]{geometry}
\KOMAoptions{headwidth=1.1\textwidth,footwidth=1.1\textwidth}
\usepackage{blindtext}
 
\pagestyle{scrheadings}
 
\ohead[\headmark]{\headmark}
\ofoot[\pagemark]{\pagemark}
\ifoot{ifoot} % inner foot
\ihead{ihead} % inner head
\cfoot{cfoot} % center foot
\chead{chead} % center head

\definecolor{hellgelb}{rgb}{1,1,0.8}
\definecolor{lightgelb}{rgb}{1,1,0.8}
\definecolor{colKeys}{rgb}{0,0,1}
\definecolor{colIdentifier}{rgb}{0,0,0}
\definecolor{colComments}{rgb}{1,0,0}
\definecolor{colString}{rgb}{0,0.5,0}

\usepackage{textcomp}
\lstset{%
	language=Lisp,%	
    float=hbp,%
    basicstyle=\ttfamily, % \footnotesize
    identifierstyle=\color{colIdentifier}, %
    keywordstyle=\color{colKeys}, %
    stringstyle=\color{colString}, %
    commentstyle=\color{colComments}, %
    alsoletter={\_},
	language= {Python},%
    columns=flexible, %
    tabsize=2, %
    morekeywords={},%
    frame=single, %
    extendedchars=true, %
    showspaces=false, %
    showstringspaces=false, %
    numbers=left, %
    numberstyle=\tiny, %
    upquote=true,
    breaklines=true, %
    backgroundcolor=\color{yellow!15}, %
    breakautoindent=true, %
    captionpos=b%
}

\lstset{literate=%
    {Ö}{{\"O}}1
    {Ä}{{\"A}}1
    {Ü}{{\"U}}1
    {ß}{{\ss}}1
    {ü}{{\"u}}1
    {ä}{{\"a}}1
    {ö}{{\"o}}1
    {~}{{\textasciitilde}}1
}

\title{Einführung in Emacs}
\author{Uwe Ziegenhagen}

\usepackage{hyperref}
\usepackage{hyperxmp}
\hypersetup{%
   pdftitle={Einführung in Emacs},
   pdfauthor={Uwe Ziegenhagen},
   pdfcopyright={Copyright (C) 2017, Uwe Ziegenhagen},
   pdfsubject={Einführung in Emacs},
   pdfkeywords={Emacs},%ggf. anpassen
   pdflicenseurl={},
   pdfcaptionwriter={},
   pdfcontactaddress={},
   pdfcontactcity={Cologne},
   pdfcontactpostcode={},
   pdfcontactcountry={Germany},
   pdfcontactphone={},
   pdfcontactemail={ziegenhagen@gmail.com},
   pdfcontacturl={http://www.uweziegenhagen.de},
   pdflang={de},
   pdfmetalang={de},
    colorlinks=true,             
    linkcolor=blue,
    filecolor=cyan,
    citecolor=green,
    urlcolor=magenta
}


\usepackage[]{tcolorbox}
%\usepackage[]{menukeys}

%\usepackage[style=authortitle-dw,backend=bibtex8]{biblatex}%authortitle-icomp
\usepackage[style=authortitle-icomp,backend=biber]{biblatex}%
\usepackage[babel,german=quotes]{csquotes}%guillemets

\addbibresource{Referenzen.bib}

\begin{document}


\maketitle

\frontmatter

\tableofcontents


\chapter*{Vorwort}

Der vorliegende Text ist aus der Erkenntnis entstanden, dass es anscheinend kein aktuelles Emacs-Buch gibt, das die aktuellen Entwicklungen im Emacs-Universum behandelt. Beschreibungen von \texttt{use-package} und anderen Paketen finden sich oft nur online und meist auch nur auf Englisch. 

Das Skript soll daher eine Einführung in Emacs und nützliche Emacs-Pakete geben. Der Fokus liegt dabei auf Windows, da ich mit Emacs hauptsächlich unter Windows arbeite,  sofern möglich werden aber auch die entsprechenden Hinweise für Linux und Mac OS X gegeben. 

\section*{Konventionen}

Folgende Konventionen gelten für den Text:

\textit{Kursiv} Neue Begriffe, Dateinamen und Dateierweiterungen werden \textit{kursiv} dargestellt.

\texttt{nichtproportional} Nichtproportionale Schrift, also \enquote{Schreibmaschinenschrift}, wird für Code-Listings benutzt.

\texttt{\bfseries nichtproportional fett} Nichtproportionale fette Schrift wird für Befehle genutzt, die der Leser über die Tastatur eingibt.

\texttt{\textit{nichtproportional kursiv}} Nichtproportionale kursive Schrift wird für Werte genutzt, die vom Nutzer vorzugeben sind oder für Werte, die sich aus dem Kontext ergeben.

%\keys{Strg + x}

Zusatzinformationen, die mit Emacs nur am Rand zu tun haben, werden in grauen Boxen dargestellt.


\section*{Die Technik hinter dem Text}

Dieses Buch ist in \LaTeX\ gesetzt worden, dem freien Satzprogramm, das ich jedem ans Herz legen möchte, der effizient komplexe oder längere Texte verfässt. Zur Verwaltung der Dateien nutze ich github \url{https://www.github.com}), das ich mittels \textit{TortoiseSVN} bediene. 

Als Schriftart wird die freie Libertine-Schrift genutzt.

\LKeyTab

\mainmatter

\chapter{Geschichtliches}

Emacs, die Abkürzung steht für \enquote{Editor MACroS}, war ursprünglich kein eigenständiger Editor, sondern lediglich eine Sammlung von Makros für TECO. TECO, was ursprünglich \enquote{\textbf{T}ape \textbf{E}ditor and \textbf{CO}rrector} stand, später jedoch zu \enquote{\textbf{T}ext \textbf{E}ditor and \textbf{CO}rrector} wurde, wurde 1962/63 als Editor für Lochstreifen entwickelt. Die Bedienung von TECO war dabei nicht auf Benutzerfreundlichkeit ausgelegt, es galt viel mehr, mit möglichst wenigen Tastendrücken den Editor zu steuern. Ebenso wie Emacs gibt es auch TECO noch heute, unter \url{http://almy.us/teco.html} kann der interessierte Leser Binaries für Windows, Linux und Mac OS~X herunterladen. 

1972 begann dann die Entwicklung von Emacs\footnote{siehe \url{https://www.emacswiki.org/emacs/EmacsHistory}} am MIT\footnote{Massachusetts Institute of Technology}, als Carl Mikkelson TECO um Funktionen zur Anzeige von Textänderungen auf dem Bildschirm erweiterte. Was heute als selbstverständlich gilt, war damals keineswegs so; es gab sogar Editoren, die die Ergebnisse der Editier-Operationen auf einem angeschlossenen Drucker ausgaben.

Im Jahr 1974 schuf Richard Stallman dann die Möglichkeit, Makros in TECO auszuführen, diese Möglichkeit wurde von den TECO-Nutzern am MIT auch rege genutzt. Die erstellten Makros wurden von Richard Stallman 1976 gesammelt und um Funktionen zur Selbstdokumentation und Erweiterbarkeit ergänzt. TECOEmacs wurde anschließend zum Standard-Editor auf den ITS-Systemen\footnote{\enquote{Incompatible TimeSharing System}} Maschinen des MIT.

In den folgenden Jahren entstanden verschiedene Emacs-Derivate, von denen MulticsEmacs von Bernard Greenberg erwähnenswert ist. MultiEmacs war in LISP geschrieben, auch Erweiterungen waren in LISP geschrieben. Die Wahl von LISP bot einfachere Erweiterungsmöglichkeiten als je zuvor und wurde daher von den meisten folgenden Emacs-Generationen genutzt. 

1984 erblickte GNU Emacs das Licht der Welt, als erstes Produkt der GNU Software Foundation. Geschrieben von Richard Stallman in C nutzte er EmacsLisp als Sprache für Erweiterungen. Die erste weithin verfügbare Version war dann Emacs 15.34, die 1985 erschien und bald der Standard für Emacs unter Unix war.

\begin{tcolorbox}[title={Wissen: Richard Stallman und die GNU Foundation},arc=0pt]
dfsfs

\end{tcolorbox}


Heute, im Jahr 2017, ist Emacs in Version 25.2 angekommen, diese Version ist auch die Basis für dieses Skript.

Einige Worte noch zum immerwährenden Kampf zwischen Emacs und VI/VIM: Wenngleich ich auch Emacs jedem VI bzw. VIM vorziehe (sonst würde dieses Skript letztere behandeln), so sind Grundkenntnisse in VI oder ED (dem Vorgänger von VI) ratsam. Im Appendix findet sich daher ein kurzer Abriss zu den wichtigsten Funktionen.

\chapter{Grundlagen}

\section{Installation}

https://www.gnu.org/software/emacs/download.html

Was ist enthalten?

Dateistruktur


\chapter{Grundlagen der Bedienung}

\section{Das eingebaute Hilfesystem}

Das eingebaute Lernprogramm

\section{Laden und Speichern}

\section{Bewegen innerhalb der Datei und zwischen den Buffern}

\section{Suchen und Ersetzen}

\chapter{Konfiguration}

\section{Manuelle Konfiguration von Paketen}

\section{Das Emacs-Paketsystem}

\section{use-package}

\chapter{Org Mode}

\chapter{Auc\TeX}

\chapter{Was sonst noch geht\ldots}

\section{ccrypt}

\section{GraphViz}

\section{Emacs als E-Mail-Programm}

\subsection{Emacs als RSS-Reader}

\section{Sweave}

\section{Spielen im Emacs}

\chapter{Programmierung}

\subsection{Das \enquote{Hello Emacs}-Beispiel}




\chapter{Weitere nützliche Software-Tools}

In diesem Kapitel sollen weitere nützliche Software-Tools vorgestellt werden, die im Zusammenhang mit Emacs interessant sind und von denen man zumindest gehört haben sollte. Denn \enquote{Hat man nur einen Hammer, so sieht alles wie ein Nagel aus!}	

Ergänzen oder ersetzen

\section{\TeX/\LaTeX}

\subsection{Geschichtliches}



\subsection{Das erste \LaTeX-Dokument}


\subsection{Eine Musterpräsentation mit \texttt{Beamer}}


\subsection{Ein Musterbrief mit \texttt{scrlttr2}}


\subsection{Wer mehr wissen möchte\ldots}


Tabellensatz eigenes Buch, daher nur kurzer Überblick

Installiere TeX Live oder MikTeX, stelle sicher dass im Pfad vorhanden

Probiere das folgende Dokument aus und übersetze es auf der Kommandozeile.

Schnapp Dir ein Buch.

\section{sed und awk}

\section{VI(M)}



\section{Python}

\backmatter

\blindtext

\begin{lstlisting}[basicstyle=\ttfamily] % <--- here
;; distraction-free
;;  https://nickhigham.wordpress.com/2016/01/14/distraction-free-editing-with-emacs/
(scroll-bar-mode 0)    ; Turn off scrollbars.
(tool-bar-mode 0)      ; Turn off toolbars.
(fringe-mode 0)        ; Turn off left and right fringe cols.
(menu-bar-mode 0)      ; Turn off menus.
;; bind fullscreen toggle to f9 key
(global-set-key (kbd "<f9>") 'toggle-frame-fullscreen)

;; http://emacs.stackexchange.com/questions/2999/how-to-maximize-my-emacs-frame-on-start-up
;; Start fullscreen (cross-platf)
(add-hook 'window-setup-hook 'toggle-frame-fullscreen t)
; emacs-doctor.com/emacs-strip-tease.html
;; Prevent the cursor from blinking
(blink-cursor-mode 0)
;; Don't use messages that you don't read
(setq initial-scratch-message "")
(setq inhibit-startup-message t)
\end{lstlisting}

\nocite{*}

\printbibliography 


\end{document}