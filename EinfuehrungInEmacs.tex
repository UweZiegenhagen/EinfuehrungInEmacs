%!TEX TS-program = Arara
% arara: pdflatex
% arara: biber
% arara: pdflatex
% arara: pdflatex

\documentclass[12pt,ngerman]{scrbook}



\usepackage[utf8]{inputenc}
\usepackage[T1]{fontenc}
\usepackage{booktabs}
\usepackage{babel}
\usepackage{graphicx}
\usepackage{paralist}
\usepackage{xcolor}

\usepackage{listings}
\usepackage{palatino}
\usepackage{sourcecodepro}
\usepackage{blindtext}

\setlength{\parindent}{0pt}
\setlength{\parskip}{1em}

\usepackage{xcolor}
\usepackage{mdframed}
\usepackage{tikz}

\usepackage[headsepline=0.5pt,footsepline=0.5pt]{scrlayer-scrpage}
\usepackage[left=2cm,right=4cm]{geometry}
\KOMAoptions{headwidth=1.1\textwidth,footwidth=1.1\textwidth}
\usepackage{blindtext}
 
\pagestyle{scrheadings}
 
\ohead[\headmark]{\headmark}
\ofoot[\pagemark]{\pagemark}
\ifoot{ifoot} % inner foot
\ihead{ihead} % inner head
\cfoot{cfoot} % center foot
\chead{chead} % center head

\definecolor{hellgelb}{rgb}{1,1,0.8}
\definecolor{lightgelb}{rgb}{1,1,0.8}
\definecolor{colKeys}{rgb}{0,0,1}
\definecolor{colIdentifier}{rgb}{0,0,0}
\definecolor{colComments}{rgb}{1,0,0}
\definecolor{colString}{rgb}{0,0.5,0}

\usepackage{textcomp}
\lstset{%
	language=Lisp,%	
    float=hbp,%
    basicstyle=\ttfamily, % \footnotesize
    identifierstyle=\color{colIdentifier}, %
    keywordstyle=\color{colKeys}, %
    stringstyle=\color{colString}, %
    commentstyle=\color{colComments}, %
    alsoletter={\_},
	language= {Python},%
    columns=flexible, %
    tabsize=2, %
    morekeywords={},%
    frame=single, %
    extendedchars=true, %
    showspaces=false, %
    showstringspaces=false, %
    numbers=left, %
    numberstyle=\tiny, %
    upquote=true,
    breaklines=true, %
    backgroundcolor=\color{yellow!15}, %
    breakautoindent=true, %
    captionpos=b%
}

\lstset{literate=%
    {Ö}{{\"O}}1
    {Ä}{{\"A}}1
    {Ü}{{\"U}}1
    {ß}{{\ss}}1
    {ü}{{\"u}}1
    {ä}{{\"a}}1
    {ö}{{\"o}}1
    {~}{{\textasciitilde}}1
}

\title{Einführung in Emacs}
\author{Uwe Ziegenhagen}

\usepackage{hyperref}
\usepackage{hyperxmp}
\hypersetup{%
   pdftitle={Einführung in Emacs},
   pdfauthor={Uwe Ziegenhagen},
   pdfcopyright={Copyright (C) 2017, Uwe Ziegenhagen},
   pdfsubject={Einführung in Emacs},
   pdfkeywords={Emacs},%ggf. anpassen
   pdflicenseurl={},
   pdfcaptionwriter={},
   pdfcontactaddress={},
   pdfcontactcity={Cologne},
   pdfcontactpostcode={},
   pdfcontactcountry={Germany},
   pdfcontactphone={},
   pdfcontactemail={ziegenhagen@gmail.com},
   pdfcontacturl={http://www.uweziegenhagen.de},
   pdflang={de},
   pdfmetalang={de},
    colorlinks=true,             
    linkcolor=blue,
    filecolor=cyan,
    citecolor=green,
    urlcolor=magenta
}


%\usepackage[style=authortitle-dw,backend=bibtex8]{biblatex}%authortitle-icomp
\usepackage[style=authortitle-icomp,backend=biber]{biblatex}%
\usepackage[babel,german=quotes]{csquotes}%guillemets

\addbibresource{Referenzen.bib}

\begin{document}


\maketitle

\frontmatter

\tableofcontents


\chapter*{Vorwort}

Das vorliegende Skript soll eine Einführung in Emacs geben.

Fokus auf Windows, die allermeisten Punkte lassen sich jedoch 1:1 auf andere Betriebssysteme oder Emacs-Varianten übertragen.

\mainmatter


\chapter{Grundlagen}

\section{Geschichtliches}

kein einzelner Editor, Familie von Editoren. 

eierlegende Wollmilchsau

Richard Stallman, GNU Software Foundation

GNU Emacs

\section{Emacs versus VI}



\section{Installation}

https://www.gnu.org/software/emacs/download.html

\chapter{Grundlagen der Bedienung}

\section{Das eingebaute Hilfesystem}

Das eingebaute Lernprogramm

\section{Laden und Speichern}

\section{Bewegen innerhalb der Datei und zwischen den Buffern}

\section{Suchen und Ersetzen}

\chapter{Konfiguration}

\section{Manuelle Konfiguration von Paketen}

\section{Das Emacs-Paketsystem}

\section{use-package}

\chapter{Org Mode}

\chapter{Auc\TeX}


\chapter{Was sonst noch geht\ldots}

\section{ccrypt}

\section{GraphViz}

\section{Emacs als E-Mail-Programm}

\subsection{Emacs als RSS-Reader}



\section{Sweave}

\section{Spielen im Emacs}

\chapter{Programmierung}

\subsection{Das \enquote{Hello Emacs}-Beispiel}




\chapter{Weitere nützliche Software-Tools}

In diesem Kapitel sollen weitere nützliche Software-Tools vorgestellt werden, die im Zusammenhang mit Emacs interessant sind und von denen man zumindest gehört haben sollte. Denn \enquote{Hat man nur einen Hammer, so sieht alles wie ein Nagel aus!}	

Ergänzen oder ersetzen

\section{\TeX/\LaTeX}

\subsection{Geschichtliches}



\subsection{Das erste \LaTeX-Dokument}


\subsection{Eine Musterpräsentation mit \texttt{Beamer}}


\subsection{Ein Musterbrief mit \texttt{scrlttr2}}


\subsection{Wer mehr wissen möchte\ldots}


Tabellensatz eigenes Buch, daher nur kurzer Überblick

Installiere TeX Live oder MikTeX, stelle sicher dass im Pfad vorhanden

Probiere das folgende Dokument aus und übersetze es auf der Kommandozeile.

Schnapp Dir ein Buch.

\section{sed und awk}

\section{VI(M)}



\section{Python}

\backmatter

\blindtext

\begin{lstlisting}[basicstyle=\ttfamily] % <--- here
;; distraction-free
;;  https://nickhigham.wordpress.com/2016/01/14/distraction-free-editing-with-emacs/
(scroll-bar-mode 0)    ; Turn off scrollbars.
(tool-bar-mode 0)      ; Turn off toolbars.
(fringe-mode 0)        ; Turn off left and right fringe cols.
(menu-bar-mode 0)      ; Turn off menus.
;; bind fullscreen toggle to f9 key
(global-set-key (kbd "<f9>") 'toggle-frame-fullscreen)

;; http://emacs.stackexchange.com/questions/2999/how-to-maximize-my-emacs-frame-on-start-up
;; Start fullscreen (cross-platf)
(add-hook 'window-setup-hook 'toggle-frame-fullscreen t)
; emacs-doctor.com/emacs-strip-tease.html
;; Prevent the cursor from blinking
(blink-cursor-mode 0)
;; Don't use messages that you don't read
(setq initial-scratch-message "")
(setq inhibit-startup-message t)
\end{lstlisting}

\nocite{*}

\printbibliography 


\end{document}